% FOAM: A Pluralistic Architecture for Explainable and Contestable AI
% FAccT 2026 Submission - REDUCED VERSION

\documentclass[manuscript,screen,review,anonymous]{acmart}

% For submission - remove these lines for camera-ready
\setcopyright{none}
\settopmatter{printfolios=true}

% Fix font expansion issue
\usepackage{lmodern}

% Package imports
\usepackage{booktabs}
\usepackage{graphicx}
\usepackage{subcaption}
\usepackage{xcolor}
\usepackage{hyperref}
\usepackage{cleveref}
\usepackage{xspace}

% Custom commands
\newcommand{\foam}{\textsc{FOAM}}
\newcommand{\ie}{i.e.,\xspace}
\newcommand{\eg}{e.g.,\xspace}

% Title and authors
\title{Framework for Openly Augmented Mediation (FOAM): A Pluralistic Architecture for Explainable and Contestable AI}

\author{Devin Gonier}
\affiliation{
  \institution{DebaterHub}
  \country{USA}
}
\email{dgonier@debaterhub.com}

\author{John Hines}
\affiliation{
  \institution{DebaterHub}
  \country{USA}
}
\email{jhines@debaterhub.com}

\author{P. Anand Rao}
\affiliation{
  \institution{University of Mary Washington}
  \department{Center for AI and the Liberal Arts}
  \country{USA}
}
\email{prao@umw.edu}

% Keywords
\keywords{Algorithmic accountability; Contestable AI; Explainable AI (XAI); Multi-agent deliberation; Evidence provenance}

\begin{document}

\begin{abstract}
High-stakes AI systems increasingly mediate access to credit, healthcare, and public benefits, yet affected parties often cannot see why a decision was made or meaningfully contest it. Even post hoc review of chain-of-thought traces from individual models can be incomplete or strategically misleading, thereby limiting accountability. We propose \foam{} (Framework for Openly Augmented Mediation), a pluralistic architecture that treats explanation as a \emph{deliberative process} rather than post-hoc narration. \foam{} instantiates differentiated agents with explicit value commitments, structures their interaction through cross-examination and rebuttal protocols, and outputs not just a recommendation but a \emph{structured record designed to support downstream contestation and review}: claims linked to sentence-level evidence provenance, surviving objections, and explicit points of disagreement. We evaluate \foam{} in evidence-grounded policy debate generation, a domain where arguments must withstand adversarial scrutiny. In a double-blind tournament of 66 cases, \foam{} outperforms human-expert and zero-shot baselines on overall quality (73.5 vs.\ 62.4 vs.\ 46.3) while achieving dramatically higher evidence verifiability (76.2\% case-level full validation vs.\ 8.7\% and 0\%). These results demonstrate that pluralistic deliberation can produce outputs that are simultaneously persuasive \emph{and} auditable, a necessary condition for contestable AI by design.
\end{abstract}

\maketitle

% Include sections - VERSION 2 WITH CUTS
% Section 1: Introduction
% Last updated from: v1_2025-01-12_baseline.md

\section{Introduction}
\label{sec:introduction}

\subsection{Accountability gap in high-stakes AI}
\label{sec:accountability-gap}

AI systems are now routinely embedded in high-stakes decision workflows---healthcare triage and documentation~\cite{obermeyer2019dissecting}, hiring and workplace management~\cite{raghavan2020mitigating}, credit and insurance~\cite{kozodoi2022fairness}, public benefits~\cite{eubanks2018automating}, and criminal-legal risk assessments~\cite{angwin2016machinebias}. In these settings, ``performance'' cannot be reduced to predictive accuracy or user satisfaction: when a system's output influences outcomes that materially affect people's rights, opportunities, or safety, \textbf{accountability requires (i) intelligible reasons and (ii) effective avenues to challenge and revise those reasons}. Yet most deployed AI remains organized around a monolithic model that produces a single authoritative output, with limited transparency into \emph{why} it said what it said and little procedural support for contesting it when it is wrong, biased, or normatively inappropriate.

This accountability gap has two tightly coupled dimensions. \textbf{Explainability} is often treated as a documentation problem---generate a rationale, a summary, or a list of features---rather than a \emph{reason-giving} problem grounded in the kinds of explanations different stakeholders actually need (\eg diagnostic vs.\ role-based explanations)~\cite{miller2019explanation,yao2024explanatory}. \textbf{Contestability}, meanwhile, is frequently bolted on as an afterthought (appeals processes, ``report a problem'' buttons, or generic feedback loops) rather than built into the architecture of reasoning itself. Meaningful contestability requires at least (a) visibility into decision logic, (b) comprehensibility for affected parties, and (c) actionable mechanisms for challenge and revision~\cite{alfrink2023contestable}. A system that cannot surface its operative assumptions, show its evidentiary basis, and support structured disagreement cannot plausibly satisfy these conditions---especially in domains where reasonable stakeholders legitimately disagree about values, tradeoffs, and acceptable risk.

\subsection{Why post-hoc ``explanations'' break: the faithfulness problem}
\label{sec:faithfulness-problem}

A central reason current explainability tooling struggles is that it frequently relies on \textbf{post-hoc self-explanation from the same model that produced the decision}. For large language models in particular, chain-of-thought and rationale-style explanations can be fluent and persuasive while remaining weakly coupled to what actually drove the output. Chen et al.\ benchmark state-of-the-art reasoning models and report low overall faithfulness scores---\eg \textbf{25\% for Claude 3.7 Sonnet and 39\% for DeepSeek R1} under their evaluation design---highlighting that models may omit or misrepresent key determinants of their answers even when explicitly prompted to ``show their work''~\cite{chen2025reasoning}. Related work similarly emphasizes that CoT can be misleading as an interpretability proxy, especially when users treat it as a reliable window into computation rather than a generated text artifact~\cite{turpin2023language}.

This ``faithfulness gap'' creates a direct accountability failure mode: if the explanation channel can drift from the decision channel, then transparency becomes performative---useful for persuasion, but unreliable for oversight, auditing, or recourse. In high-stakes contexts, that is not a subtle limitation; it is a design-level mismatch between what institutions need (verifiable reasons and traceable evidence) and what monolithic systems can robustly provide. The core implication is architectural: \textbf{if we want explanations that can support contestation, we need systems that can produce multiple, checkable reason-giving traces---not a single narrative generated by the same mechanism being explained.} This motivates pluralistic approaches that externalize disagreement, force explicit warrants, and attach provenance to claims so that challenges can target the actual moving parts of the reasoning.

\subsection{What we propose (FOAM) and what is new}
\label{sec:foam-intro}

This paper develops and evaluates \textbf{pluralistic AI systems} that operationalize explainability and contestability through \textbf{structured multi-agent deliberation} rather than post-hoc narration. We introduce \textbf{\foam{} (Framework for Openly Augmented Mediation)}, an architecture that treats accountable AI outputs as the product of a mediated process:
\begin{enumerate}
    \item \textbf{Differentiated agents} with distinct roles and epistemic commitments (\eg advocate, skeptic, evidence-checker, values/impact assessor),
    \item \textbf{Deliberative protocols} that require agents to advance and respond to claims under explicit constraints (\eg argument typing, cross-examination, and structured rebuttal), and
    \item \textbf{Sublation operators}---formal mechanisms for preserving what survives critique while revising what fails, so that the system's final output is not merely an average of perspectives but a documented transformation through contestation.
\end{enumerate}

The intended artifact is not just a recommendation, but a contestable record: claims, counterclaims, evidentiary supports, explicit points of disagreement, and the rationale for any resolution.

We make three contributions:
\begin{enumerate}
    \item \textbf{Framework:} we provide a unified account of explainability \emph{and} contestability as a single design target, arguing that they should be treated jointly and realized through pluralistic mediation rather than monolithic self-report.
    \item \textbf{Architecture and mechanisms:} we formalize \foam{} as an implementable blueprint---agents, protocols, and revision operators---paired with provenance-oriented design choices that make challenges actionable (\eg grounding claims in checkable evidence rather than free-form summarization).
    \item \textbf{Empirical validation:} we report results from an evaluation of pluralistic debate generation in a double-blind tournament of \textbf{66 policy debate cases}, where our structured multi-agent system achieved an overall score of \textbf{81.7} compared to \textbf{70.1} for human experts and \textbf{50.6} for zero-shot AI, while also achieving \textbf{76.2\%} perfect evidence validation compared to \textbf{8.7\%} for human experts and \textbf{0\%} for unstructured AI---demonstrating that pluralistic architectures can produce outputs that are simultaneously more persuasive \emph{and} more verifiable in an adversarial, evidence-sensitive setting.
\end{enumerate}

We close by discussing implications for AI governance and by outlining a research agenda for \textbf{contestable AI by design}.

% Section 2: Related Work
% Last updated from: v1_2025-01-12_baseline.md

\section{Accountability requirements and related work}
\label{sec:related-work}

\subsection{Explainability requirements beyond transparency}
\label{sec:explainability-requirements}

Contemporary calls for ``explainable AI'' often conflate \textbf{transparency} (exposing internal mechanisms) with \textbf{explanation} (providing reasons that are meaningful for a particular audience and purpose). Lipton argues that interpretability is not a single property and that many ``explanations'' in ML function as \emph{post-hoc rationalizations} whose relationship to actual model behavior is ambiguous, especially when the explanation's audience is a regulator, decision-subject, or domain expert rather than a model developer~\cite{lipton2018mythos}. Relatedly, Doshi-Velez \& Kim emphasize that interpretability claims must be made relative to \textbf{use context}---including the user's expertise, the stakes, and the kind of decision being supported---because what counts as a satisfactory explanation differs across settings~\cite{doshivelez2017towards}. In high-stakes domains, this motivates either (i) models that are inherently interpretable, or (ii) explanation mechanisms that achieve a comparable standard of \emph{reliability and auditability} rather than superficial plausibility~\cite{rudin2019stop}.

For accountability, explanations must be more than persuasive narratives; they must be \textbf{diagnostically useful} and \textbf{robust to strategic manipulation}. The NLP interpretability literature distinguishes \emph{plausibility} (does an explanation look reasonable?) from \emph{faithfulness} (does it track the true basis of the output?), arguing that faithful explanations require evaluation criteria beyond ``nice-sounding'' rationales~\cite{jacovi2020towards}. Explainability requirements in FAccT-relevant deployments should be stated in terms of \textbf{checkability}: tracing claims to concrete support and isolating points of disagreement~\cite{miller2019explanation,jacovi2020towards}.

\subsection{Contestability as a system property}
\label{sec:contestability}

Explainability alone does not guarantee that affected parties can meaningfully challenge an AI-mediated decision; contestability is best treated as a \textbf{system-level governance property} rather than an after-the-fact user interface feature. Alfrink et al.\ frame ``contestable AI by design'' as the view that systems should be built to \emph{support} contestation---through traceability, structured justification, and pathways for challenge---rather than treating contestation as an external legal or organizational process that happens ``around'' the model~\cite{alfrink2023contestable}. Legal scholarship on automated decision-making similarly emphasizes that accountability requires more than disclosure: decision-subjects need procedures to \emph{question, rebut, and obtain redress}, and these procedures depend on the availability of intelligible grounds and records of how outputs were produced~\cite{kroll2017accountable}. This is particularly important because the existence and scope of a freestanding ``right to explanation'' under the GDPR is contested, with influential analyses arguing that GDPR does not straightforwardly provide a general right to detailed model explanations---reinforcing the need for contestability mechanisms that do not rely on a single doctrinal reading of transparency rights~\cite{wachter2017right}.

Operationally, contestability implies three minimal requirements:
\begin{enumerate}
    \item \textbf{Visibility} that an automated or AI-assisted decision has occurred and can be challenged;
    \item \textbf{Comprehensibility} of the stated grounds and supporting materials; and
    \item \textbf{Actionability}, meaning a practical pathway to present counterevidence/counterarguments and obtain review and potential revision~\cite{alfrink2023contestable,kroll2017accountable}.
\end{enumerate}

Legal and governance frameworks reinforce this design target: GDPR Article 22 provisions and EU Trustworthy AI guidance treat accountability as including mechanisms for redress and challenge~\cite{eu2016gdpr,eu2019trustworthy}. This motivates \textbf{contestability as an end-to-end workflow} linking reasons to evidence and enabling structured challenge~\cite{raji2020closing,alfrink2023contestable}.

\subsection{Pluralistic and deliberative approaches to accountability}
\label{sec:pluralistic-approaches}

In high-stakes settings, disagreement is not merely empirical but normative. Feminist epistemology argues that knowledge claims are situated and ``view from nowhere'' objectivity can mask whose interests are operationalized~\cite{haraway1988situated}. For AI accountability, this motivates an architectural stance: systems should make \textbf{value trade-offs explicit} and preserve dissenting considerations that can be examined and contested~\cite{miller2019explanation,haraway1988situated}.

Recent work in value alignment and governance likewise emphasizes that ``alignment'' is underdetermined when stakeholders disagree about objectives, priorities, and acceptable risks. Kasirzadeh distinguishes forms of alignment that presume a single coherent value target from approaches that treat plural and conflicting values as first-class constraints---implying that accountability mechanisms must represent disagreement rather than suppress it~\cite{kasirzadeh2023conversation}. In parallel, ``society-in-the-loop'' framings argue that algorithmic systems are components of an evolving social contract and therefore require institutionalized interfaces for dispute, oversight, and revision~\cite{rahwan2018society}. In FAccT terms, these perspectives justify \textbf{pluralistic explanation}: not as an optional UX feature, but as a governance mechanism that helps stakeholders identify where the system's reasoning depends on contestable assumptions~\cite{rahwan2018society,kasirzadeh2023conversation}.

\subsection{Multi-agent deliberation and debate in AI}
\label{sec:multi-agent-deliberation}

A technical pathway to operationalizing pluralism is to replace monolithic generation with \textbf{structured multi-agent deliberation}, including debate-style protocols. In AI safety, ``debate'' was proposed as a scalable oversight mechanism in which adversarial argumentation can surface flaws or deception that a single system might otherwise hide~\cite{irving2018ai}. Subsequent theoretical work studies conditions under which debate can be made efficient and verifiable, strengthening the conceptual link between adversarial dialogue and reliable oversight~\cite{browncohen2024scalable}. Empirically, multi-agent debate among language models has been reported to improve factuality and reasoning in some settings, suggesting that disagreement and cross-examination can function as error-correction dynamics rather than mere rhetoric~\cite{du2023improving}. However, most ``LLM debate'' results are evaluated in terms of accuracy or judge preference; they do not, by themselves, guarantee that the resulting justifications are \textbf{auditable} or that third parties can meaningfully contest specific premises, evidence selections, or value judgments~\cite{rudin2019stop,jacovi2020towards}.

Computational argumentation provides complementary foundations for making deliberation outputs contestable because it supplies explicit representations of \textbf{claims, warrants, attacks, defenses, and (in value-based variants) normative priorities}. Toulmin's model remains foundational for analyzing argument structure in terms of claims supported by warrants and backing~\cite{toulmin1958uses}. Formal work in AI argumentation further develops abstract and assumption-based frameworks for representing defeasible reasoning, while value-based argumentation captures how outcomes change when different values are prioritized~\cite{benchcapon2009argumentation,toni2014tutorial}. Surveys connecting argumentation and XAI argue that these representations can support explanation as a structured object of inquiry---closer to an ``inspectable case'' than a narrative rationale---because stakeholders can contest particular premises or inference steps and observe how the conclusion changes~\cite{vassiliades2021argumentation}. This literature motivates the core related-work claim that a \emph{contestable} AI system should produce not only an answer, but also a \textbf{dispute-ready argumentative record}: reasons decomposed into contestable units, linked to supporting materials, and amenable to revision under challenge~\cite{kroll2017accountable,vassiliades2021argumentation}.

% Section 3: FOAM Approach - CC CUTS VERSION
% Cuts: ~250 words

\section{FOAM approach: pluralistic architecture for explainability and contestability}
\label{sec:foam-approach}

\subsection{Design goals and accountability threat model}
\label{sec:design-goals}

Building on Section~\ref{sec:related-work}, we treat \emph{explainability} and \emph{contestability} as properties of an \textbf{epistemic process}, not a post-hoc narrative. We introduce \textbf{\foam{} (Framework for Openly Augmented Mediation)}: a pluralistic, multi-agent architecture producing an answer \emph{plus} a structured record of how it was stress-tested and synthesized. \foam{} is organized around three primitives: (i) \emph{differentiated agents} parameterized by explicit stance data structures, (ii) \emph{deliberative protocols} forcing critique and revision, and (iii) \emph{sublation} operators that synthesize without erasing disagreement. Figure~\ref{fig:system-overview} provides a system overview.

\begin{figure}[htbp]
\centering
\includegraphics[width=\columnwidth]{figures/FOAM_system_overview.png}
\caption{\foam{} system architecture. Differentiated agents with explicit perspective nodes engage in deliberative protocols producing accountability artifacts including a consensus core, conditional claims, and dissent memo.}
\Description{Architecture diagram showing the FOAM system with six main processing stages connected by arrows, with Perspective Nodes feeding into agents, a Mediation Graph capturing deliberation, and three output types.}
\label{fig:system-overview}
\end{figure}

Our threat model assumes base generative models can (a) produce fluent but false claims (``hallucination'')~\cite{ji2023survey}, (b) rationalize decisions after the fact~\cite{turpin2023language}, (c) collapse multiple perspectives into a dominant frame, and (d) bury value tradeoffs inside unstructured prose. \foam{}'s core design makes \emph{points of potential failure} explicitly addressable: disagreements are surfaced, objections are first-class objects, and synthesis preserves traceability from contested premises to recommendations.

\subsection{Differentiated agents via explicit perspective representation}
\label{sec:differentiated-agents}

\foam{} instantiates agents each assigned an explicit \emph{Perspective Node} encoding \emph{who the agent is epistemically}---domain role, value priorities, and reasoning schema. This implements ``situated'' explanation in an auditable way: the system discloses positions and enables critique of \emph{perspective selection} itself~\cite{haraway1988situated}. Perspective nodes are operational constraints shaping what evidence is legitimate, which impacts are foregrounded, and which argument schemes are preferred.

A perspective node has three components: (1) \textbf{role} (\eg regulator, clinician, community advocate), (2) \textbf{normative weighting} (\eg safety vs autonomy vs equity), and (3) \textbf{epistemic policy} (\eg acceptable support standards). During deliberation, \foam{} enforces \emph{stance coherence}: if generated warrants contradict the declared stance, the system flags the inconsistency.

Perspective nodes enable \textbf{second-order contestation}: stakeholders can dispute not only conclusions, but the \emph{legitimacy of the perspective configuration} (\eg ``Why is utilitarian cost-effectiveness in scope here?''). \foam{} makes the stance set an explicit input and target for governance~\cite{kasirzadeh2024plurality}. This means \foam{} can be rerun with added perspectives, reweighted priorities, or altered evidentiary rules, producing \emph{comparative, contestable} outcomes.

\subsection{Deliberative protocol: dialectical refinement and mediation trace}
\label{sec:deliberative-protocol}

\foam{}'s deliberation is a \textbf{mediation loop}: (1) \emph{seeding} (instantiate agents + perspectives), (2) \emph{local drafting} (independent proposals), (3) \emph{cross-examination} (structured objections), (4) \emph{evaluation} (scoring draft--objection pairs), and (5) \emph{revision + synthesis}. The accountability point: \textbf{deliberation guarantees structured opportunities to find and localize error}, and records what happened when error was raised.

Cross-examination produces a \textbf{mediation graph}: a trace linking \emph{which agent} made \emph{which claim}, what objections were raised, how claims were revised, and which survived. This is the audit primitive: stakeholders can point to \emph{the specific node} where they disagree. The trace can be expressed using standard provenance representations (\eg PROV-O)~\cite{lebo2013prov}.

\subsection{Sublation: synthesis without erasure}
\label{sec:sublation}

After critique, \foam{} applies a \textbf{sublation operator}: synthesis preserving what is valuable in competing positions while retaining unresolved tensions. Synthesis is disallowed from silently discarding material objections or collapsing incompatible frames into unmarked compromise. Sublation emits three artifacts: \textbf{a consensus core} (claims surviving cross-stance critique), \textbf{conditional claims} (branching on unresolved priorities), and a \textbf{dissent memo} (recording conflicts and contested premises).

\subsection{Inspectable argument structure: Toulmin decomposition and typed syllogisms}
\label{sec:argument-structure}

To make contestation actionable, \foam{} constrains outputs into \textbf{inspectable argument structure}. We adopt Toulmin-style decomposition---claim, grounds, warrant, backing, qualifier, rebuttal---because it maps to ``what can be challenged'': stakeholders can contest evidence, the inferential link, scope conditions, or missing counterevidence~\cite{toulmin1958uses,vassiliades2021argumentation}.

\foam{} employs \textbf{typed syllogisms}---argument templates enforcing completeness (\eg Advantage = Uniqueness + Link + Impact). These function as contestability scaffolds: if a stakeholder disputes the conclusion, the system points to the \emph{specific weak component}, and the mediation graph shows whether it was raised in critique~\cite{snider2008code}.

Template tree traversal operationalizes structural contestability. At each branch point, the system records which template was selected (\eg ``traditional 1AC'' vs.\ ``kritik''), what resource allocation was applied, and whether novel templates were generated. Stakeholders can dispute not only \emph{what} claims were made, but \emph{why the structure took this form}. Unlike chain-of-thought where reasoning and response are interwoven, template traversal is a discrete prior step serving as foundational infrastructure to drafting.

% Section 4: Case Study
% Last updated from: v1_2025-01-12_baseline.md

\section{Case study system: evidence-grounded policy debate generation}
\label{sec:case-study}

\subsection{Why policy debate is an accountability crucible}
\label{sec:debate-crucible}

We instantiate \foam{} in a domain where \emph{contestability is native to the task}: American competitive policy debate. Policy debate is a two-team adversarial format in which teams argue for and against a policy proposal under strict procedural constraints~\cite{snider2008code}. In this ecosystem, argument quality is not evaluated purely as rhetorical fluency; instead, the activity is structured around \emph{traceable evidentiary support} and explicit clash, so claims can be challenged in real time and revisited across subsequent speeches. Critically, policy debate operationalizes ``grounding'' through an established evidence artifact: the \emph{debate card}. A card typically includes (i) a short biased summary intended to support a specific argumentative function, (ii) a full citation, and (iii) verbatim quoted source text, often with token-level highlighting that marks precisely what will be read into the round. Competitive success is strongly coupled to evidence quality and its deployment, creating an evaluation environment where provenance and verifiability are not optional.

\subsection{Pipeline overview}
\label{sec:pipeline-overview}

Figure~\ref{fig:pipeline} summarizes our \textbf{five-phase pipeline} for generating an evidence-grounded constructive speech (the 1AC, in our evaluation setting). Phases 1--3 produce an inspectable argumentative plan in typed components (perspective assignment $\rightarrow$ strategic plan $\rightarrow$ template traversal), Phase 4 binds each argumentative component to \emph{verbatim evidence at sentence granularity} (sentence-level provenance), and Phase 5 compiles and verifies the result (structural conformance, evidence/claim alignment, and perspective consistency). The key design principle is to keep the model in a role where it can be audited: rather than ``write a persuasive case and cite sources,'' the system decomposes ``case construction'' into a sequence of constrained decisions that leave a machine-checkable trail.

\begin{figure}[htbp]
\centering
\includegraphics[width=\columnwidth]{figures/contestability_mechanics.png}
\caption{Five-phase pipeline with accountability mechanisms. Phases 1--3 (Perspective Assignment, Plan Generation, Template Traversal) handle argumentative planning. Phase 4 (Evidence Binding) creates sentence-level provenance by selecting specific sentence IDs rather than paraphrasing. Phase 5 (Compilation) enforces verification checks. The output is a contestable speech artifact with claims, warrants, and traceable evidence links.}
\Description{Flow diagram showing five sequential phases: Phase 1 Perspective Assignment (with 24-dim perspective node output), Phase 2 Plan Generation (with coherence scoring and dialogical refinement), Phase 3 Template Traversal (with critic, evaluator, and typed syllogisms), Phase 4 Evidence Binding (with sentence indexing and sentence-level provenance highlighted in green as accountability mechanisms), and Phase 5 Compilation (with verification checks and provenance map). All phases flow to a final Contestable Speech Artifact output containing claims, warrants, and evidence links.}
\label{fig:pipeline}
\end{figure}

\subsection{Phases 1--3: perspective assignment, planning, and template traversal}
\label{sec:phases-1-3}

Phases 1--3 produce an inspectable argumentative plan through three contestability-relevant operations. In \textbf{Phase 1}, the system assigns an explicit perspective node (Section~\ref{sec:differentiated-agents}), making the evaluative frame a first-class auditable choice. In \textbf{Phase 2}, a dialectical refinement loop stress-tests the strategic plan: a Critic agent issues typed objections (logical gap, missing evidence, value conflict, scope overreach), an Evaluator scores each objection's materiality, and the Proposer revises or rebuts. This cycle iterates at least three times, and \emph{all objections---including dismissed ones---remain in the mediation graph}, enabling downstream reviewers to inspect whether a weakness was raised and why the response was deemed adequate.

In \textbf{Phase 3}, template tree traversal expands the plan into a typed syllogism scaffold (\eg Advantage = Uniqueness + Link + Impact). At each branch point, the system records which template was selected, what word allocation was applied (\eg 30\% impact, 40\% link), and whether novel templates were generated. This trace enables a distinct class of challenges: stakeholders can dispute not only \emph{what} claims were made, but \emph{why the argumentative structure took this form rather than another}---for instance, contesting that a utilitarian impact calculus was chosen when the underlying values favor a rights-based framing.

\subsection{Phase 4: sentence-level provenance}
\label{sec:phase-4}

\textbf{Motivation.} Retrieval-augmented generation can reduce hallucinations~\cite{lewis2020rag,shuster2021retrieval}, but it does not eliminate a central accountability failure mode: models may still produce claims that are \emph{unsupported by}, \emph{in conflict with}, or \emph{misattributed to} retrieved text. Recent benchmarks explicitly document that, even under RAG setups, LLM outputs can contain unsupported or contradictory content relative to the retrieved passages~\cite{gao2023rarr}. Phase 4 therefore implements a stronger constraint than ``retrieve then paraphrase'': it forces the model to operate over \emph{sentence identifiers} rather than free-form rewriting of source material.

\textbf{Mechanism.} Phase 4 is a two-step procedure:

\textbf{Step (a): sentence indexing.} The system queries a debate-evidence store (vector database of debate ``cards'') and segments retrieved documents into sentences with stable identifiers \texttt{(document\_id, sentence\_id)}.

\textbf{Step (b): evidence selection.} The LLM selects sentence IDs supporting each argument slot and generates short ``tags'' stating what each sentence establishes. The model never restates evidence; final content is assembled from retrieved sentences directly. This eliminates fabricated quotations by construction: the model can select wrong sentences but cannot invent ones not in the retrieved set.

\textbf{Contestability properties.} Sentence-level provenance changes contestation from ``argue about what the model meant'' to ``inspect what the model relied on.'' Stakeholders can challenge (i) \emph{relevance}, (ii) \emph{adequacy}, or (iii) \emph{selection bias}---each targeting a concrete sentence ID. This aligns with policy debate norms where quoted text is the unit of disputation.

\subsection{Phase 5: compilation and verification checks}
\label{sec:phase-5}

Phase 5 compiles the typed argument scaffold (Phase 3) and the evidence bindings (Phase 4) into a final speech artifact suitable for evaluation. Compilation preserves the provenance map: each substantive claim in the rendered speech remains traceable to one or more sentence IDs plus citation metadata. The system then runs verification checks that are directly tied to the accountability requirements:
\begin{enumerate}
    \item \textbf{Structural completeness} (template validators---\eg required components are present),
    \item \textbf{Evidence/claim alignment} (each slot has at least one bound sentence; missing bindings fail closed), and
    \item \textbf{Perspective consistency} (warrants and impacts do not contradict the declared perspective node from Phase 1).
\end{enumerate}

Figure~\ref{fig:pipeline} highlights where provenance is created (Phase 4) and where it is enforced (Phase 5).

% Section 5: Evaluation
% Last updated from: v1_2025-01-12_baseline.md

\section{Empirical Evaluation}
\label{sec:evaluation}

\subsection{Research questions}
\label{sec:research-questions}

We evaluate \foam{}'s accountable-generation claims using an \emph{audit-style} design: we define explicit research questions, compare against salient baselines, and report both performance outcomes and traceability outcomes as first-class metrics. This approach aligns with established work on internal algorithmic auditing and emerging ``assurance audit'' perspectives, which emphasize that accountability requires not only outcome quality, but also artifacts and procedures that make decisions inspectable and challengeable~\cite{raji2020closing,lam2024assurance}.

We ask whether \foam{} improves:
\begin{itemize}
    \item \textbf{RQ1:} Quality/persuasiveness under adversarial evaluation
    \item \textbf{RQ2:} Evidence verifiability---a necessary precondition for contestability
    \item \textbf{RQ3:} Whether gains are attributable to the accountability mechanisms rather than model strength or corpus advantages
\end{itemize}

\textbf{Scope of evaluation.} We evaluate verifiability rather than end-to-end contestability because the latter requires human-subject studies of challenge behaviors (time-to-locate-disputed-premise, challenge success rates, revision outcomes), which we scope as future work (Section~\ref{sec:future-work}). However, policy debate provides partial ecological validity: arguments that cannot survive adversarial cross-examination are systematically punished, so tournament success functions as a domain-native stress test for whether outputs can withstand structured challenge.

We acknowledge that RQ3 is only partially addressed: while we control for prompt engineering and evidence access in baselines, we do not isolate contributions of (i) pluralistic deliberation vs (ii) sentence-level provenance vs (iii) template structure. We discuss this limitation and outline ablation designs in Section~\ref{sec:limitations}.

\subsection{Experimental design and baselines}
\label{sec:experimental-design}

\textbf{Task selection.} We evaluate in policy debate generation because it combines long-horizon argumentative planning, adversarial robustness expectations, and strict evidentiary norms~\cite{slonim2021autonomous}.

\textbf{Debate artifact.} We focus on the \textbf{first affirmative constructive (1AC)}, the most demanding generative unit: it must introduce a full strategic position under tight length constraints while maintaining evidentiary support---a strong proxy for high-stakes accountable generation.

\textbf{Corpus and baselines.} We ran a \textbf{double-blind tournament of 66 cases} drawn from three sources:
\begin{enumerate}
    \item \textbf{\foam{}-based structured system} ($n=22$), generated via differentiated perspectives, iterative dialectical refinement, typed syllogisms, and sentence-level provenance;
    \item \textbf{Human expert baseline} ($n=23$), sampled from expert-authored training materials from highly competitive policy debate programs; and
    \item \textbf{Zero-shot AI baseline} ($n=21$), produced by frontier models (Gemini/Claude/ChatGPT/Grok) using prompt engineering and web-research access but without debate-specific pluralistic architecture.
\end{enumerate}

\textbf{Baseline controls (zero-shot AI).} We generated baselines using Claude 4.5, GPT-5, SuperGrok Heavy, and Gemini 2.5 in research modes with a standardized ``mega-prompt'' enforcing the same 1AC conventions: \textbf{1300--1700 words}, debate formatting, fixed advantage/solvency structure, and a strict \textbf{no-fabrication policy} requiring models to mark uncertainty as \textbf{[EVIDENCE NEEDED]}. Unlike FOAM, baselines did not use multi-agent deliberation, typed syllogisms, or sentence-level provenance; citations remained unconstrained and were evaluated under the same validation pipeline. We generated one case per topic per condition and used outputs as-is.

\textbf{Evidence retrieval.} \foam{} combines multi-hop web research with OpenDebateEvidence lookup~\cite{roush2024opendebate}, selecting best-fit evidence per argument slot; sentence-level IDs enable traceability.

\textbf{Model separation.} To mitigate self-enhancement bias, generation uses Claude Haiku/Sonnet while judging uses Claude Opus 4. We acknowledge same-provider models may share preferences; Section~\ref{sec:limitations} discusses this limitation.

\subsection{Judging rubric and scoring}
\label{sec:judging-rubric}

\textbf{Tournament format.} All 66 submissions were anonymized and evaluated in a double-blind tournament (Appendix~\ref{sec:appendix-evaluation}). Tables~\ref{tab:main-results}--\ref{tab:validation} report aggregate statistics over the full corpus.

\textbf{Rubric and judge.} Following LLM-as-judge methodology~\cite{zheng2023judging}, a Claude Opus 4 judge evaluated each case on five weighted dimensions (Argumentation 25\%, Evidence 25\%, Coherence 20\%, Innovation 15\%, Viability 15\%; full rubric in Appendix~\ref{sec:appendix-evaluation}).

\subsection{Evidence validation methodology}
\label{sec:evidence-validation}

\textbf{Why evidence validation matters.} In contestable systems, stakeholders must locate and evaluate claim grounds. Audit frameworks emphasize that assurance depends on traceable evidence artifacts~\cite{raji2020closing,lam2024assurance}. We operationalize verifiability as a measurable property of citations.

\textbf{Automated citation checks.} Each citation was classified as exact match, partial match, paraphrase, or fabricated based on source verification (classification criteria in Appendix~\ref{sec:appendix-provenance}).

We report two complementary metrics:
\begin{itemize}
    \item \textbf{Citation-Level Exact Match Rate (CEMR):} The proportion of individual citations achieving exact or partial match status. This measures per-citation fidelity.
    \item \textbf{Case-Level Full Validation Rate (CFVR):} The proportion of cases where \textit{all} citations achieve exact or partial match. This measures whether an entire artifact is audit-ready.
\end{itemize}
Table~\ref{tab:validation} reports CFVR (the stricter, case-level metric). The key finding holds under both metrics: \foam{} dramatically outperforms baselines on evidence integrity.

\textbf{Validation procedure.} For \foam{} outputs, validation is near-trivial pointer integrity: the system assembles evidence from retrieved sentences by ID, so verification reduces to checking that the assembled text matches the indexed source at that ID. For baseline outputs, validation requires resolving citations to external sources (URLs, bibliographic references) and performing substring matching. This asymmetry partly explains \foam{}'s advantage: the architecture guarantees source accessibility by construction, whereas baselines may cite sources that are paywalled, non-digitized, or incompletely referenced.

\subsection{Results}
\label{sec:results}

\textbf{Main tournament outcomes.} Table~\ref{tab:main-results} reports aggregate performance by source. The \foam{}-based system achieved the highest overall score (\textbf{73.5}) relative to human experts (\textbf{62.4}) and zero-shot AI (\textbf{46.3}). The largest gap appears in \textbf{Evidence Quality} (\textbf{78.3} vs.\ \textbf{51.4} vs.\ \textbf{20.5}), consistent with the claim that provenance-constrained generation shifts the system from persuasive-but-unreliable outputs toward persuasive-and-grounded outputs.

\begin{table}[htbp]
\caption{Tournament Results by Source}
\label{tab:main-results}
\centering
\begin{tabular}{lccc}
\toprule
\textbf{Metric} & \textbf{\foam{}} & \textbf{Human Expert} & \textbf{Zero-shot AI} \\
\midrule
Overall Score & 73.5 & 62.4 & 46.3 \\
Evidence Quality & 78.3 & 51.4 & 20.5 \\
\bottomrule
\end{tabular}
\end{table}

\textbf{Evidence validation and verifiability.} Table~\ref{tab:validation} reports Case-Level Full Validation Rates (CFVR). \foam{} achieved \textbf{76.2\%} CFVR, compared to \textbf{8.7\%} for the human expert baseline and \textbf{0\%} for zero-shot AI. This is the central accountability result: the \foam{} pipeline does not merely produce arguments that a judge model rates as ``good,'' but produces arguments whose evidentiary support can be mechanically verified at scale.

\begin{table}[htbp]
\caption{Case-Level Full Validation Rate (CFVR): percentage of cases where ALL citations achieve exact or partial match.}
\label{tab:validation}
\centering
\begin{tabular}{lc}
\toprule
\textbf{Source} & \textbf{CFVR (\%)} \\
\midrule
\foam{} System & 76.2 \\
Human Expert & 8.7 \\
Zero-shot AI & 0.0 \\
\bottomrule
\end{tabular}
\end{table}

\begin{figure}[htbp]
\centering
\includegraphics[width=\columnwidth]{figures/combined_results.pdf}
\caption{Tournament results comparing \foam{}, human expert baselines, and zero-shot AI. (a) Overall and Evidence Quality scores. (b) Case-Level Full Validation Rate (CFVR)---the percentage of cases where all citations achieve exact or partial match with source text. \foam{} achieves 76.2\% CFVR vs.\ 8.7\% for human experts and 0\% for zero-shot AI.}
\Description{Two bar charts side by side. Left chart shows Overall Score and Evidence Quality for three sources: FOAM (73.5/78.3), Human Experts (62.4/51.4), and Zero-shot AI (46.3/20.5). Right chart shows Perfect Validation percentages: FOAM 76.2\%, Human Experts 8.7\%, Zero-shot AI 0\%.}
\label{fig:evaluation-results}
\end{figure}

\textbf{Interpreting what is doing the work.} Two mechanisms plausibly drive the observed gap: (i) \textbf{pluralistic deliberation} (multi-perspective critique and refinement) improves strategic coherence and argument coverage, while (ii) \textbf{sentence-level provenance} directly improves evidence integrity and sharply limits fabrication opportunities. Several high-scoring FOAM cases achieved perfect validation (fidelity = 1.0), indicating that high persuasive quality and high verifiability can co-occur under the \foam{} constraint regime.

% Section 6: Implications
% Last updated from: v2_2025-01-12.md

\section{Implications for accountable AI systems}
\label{sec:implications}

\foam{} reframes explanation as a structured record designed to support contestation rather than a post-hoc narrative. Instead of producing a single rationale, the system outputs (i) an auditable argument structure (claims, warrants, rebuttals), (ii) explicit perspective configurations, and (iii) sentence-level provenance linking each substantive claim to a checkable source span. This shifts accountability from ``did the explanation sound plausible?'' to ``which premises and evidence does the output depend on, and where can a challenge be lodged?''

Operationally, \foam{} supports contestation at three levels~\cite{alfrink2023contestable}: (1) \textbf{evidence disputes} (a cited sentence does not support the tagged claim; missing counterevidence), (2) \textbf{inferential disputes} (the warrant connecting evidence to conclusion is invalid or incomplete), and (3) \textbf{normative disputes} (the perspective/value configuration is illegitimate or incomplete for the context). Because these objects are explicit, a reviewer can localize disagreement to specific nodes and request revision without reopening the entire output as free-form prose.

\textbf{Template trees as contestability scaffolds.} Each typed syllogism component---uniqueness, link, impact---corresponds to a discrete contestation target with evidentiary bindings. The traversal log records \textit{what} claims were made and \textit{why} the structure took this form, enabling evidence, structural, and normative challenges at the level of epistemic assumptions, not only conclusions.

Institutionally, the resulting artifact functions as an auditable dossier that can support downstream review within existing governance workflows (internal review, incident response, assurance audits); dispute resolution itself requires institutional process beyond what the technical system provides. The technical contribution is not replacing due process, but supplying the structured, traceable materials that make procedural review feasible at scale.

% Section 7: Limitations and Future Work
% Last updated from: v1_2025-01-12_baseline.md

\section{Limitations and future work}
\label{sec:limitations}

\subsection{Methodological limitations and validity threats}
\label{sec:methodological-limitations}

First, our primary outcome measure relies on an automated judge (Claude Opus 4) to score debate artifacts under a fixed rubric. While LLM-as-judge evaluation is increasingly standard at scale, it is known to exhibit systematic biases (\eg position effects, verbosity/style sensitivity, and self-enhancement tendencies) and may be vulnerable to prompt- or framing-based perturbations that shift preferences without corresponding semantic differences~\cite{zheng2023judging,shi2024judging,chen2024humans}. We mitigate these threats through three design choices: (i) double-blinding (judge sees anonymized cases), (ii) model separation (generation uses Claude Haiku/Sonnet; judging uses Claude Opus 4, a different model that did not produce the outputs it evaluates), and (iii) pairing quality scores with an independent evidence-validation audit that does not depend on LLM judgment.

Nevertheless, models from the same provider may share systematic preferences (\eg favoring structured outputs, particular rhetorical patterns, or longer responses). The reported tournament results should be interpreted as descriptive for this evaluation setup. Future replications should triangulate across: (a) judge models from different providers (GPT-4, Gemini, open-source alternatives), (b) human expert adjudication on a representative subset, and (c) robustness analysis across rubric variations. We view the evidence validation results (Table~\ref{tab:validation}) as more robust than the quality scores (Table~\ref{tab:main-results}), since validation is computed deterministically without LLM judgment.

Second, our system's accountability guarantees are conditioned on the properties of the underlying evidence substrate. Sentence-level provenance constrains the model to point to specific source sentences rather than inventing citations, but it does not ensure that the retrieved evidence is complete, representative, or up to date. Coverage gaps, topical skew, and retrieval errors can shape which arguments are discoverable, and can yield outputs that are ``well-cited'' yet misleading due to selection effects, over-aggregation, or missing context~\cite{roush2025superpersuasive}. These concerns are not unique to debate generation: any contestability mechanism built on curated corpora inherits the corpus' blind spots. Accordingly, \foam{} should be viewed as an approach to making claims auditable and challengeable---not as a guarantee that the selected evidence is normatively ``best'' or epistemically sufficient.

Third, our evaluation scope is intentionally narrow and therefore limits external validity. We benchmark a specialized argumentative domain (policy debate) and a bounded artifact type (constructive case generation), and we do not yet measure downstream stakeholder contestation behaviors (\eg whether affected parties can efficiently detect, understand, and successfully challenge specific warrants or citations). Additionally, our CFVR metric is strict by design: it favors verbatim traceability and can under-credit faithful paraphrase or correct claims supported by multiple dispersed sentences. Conversely, the metric may fail to detect other fidelity failures (\eg cherry-picked quoting or context stripping) that require richer contextual checks. These are appropriate trade-offs for an audit-style evaluation, but they motivate follow-on studies with complementary human-centered and context-sensitive validation protocols.

Fourth, our evaluation embeds cultural assumptions. American policy debate reflects adversarial norms that may not map to all contestation contexts; in settings where parties lack advocacy resources, adversarial framing may exacerbate power asymmetries. Future work should explore collaborative deliberation instantiations (\eg citizen assemblies).

\subsection{Safety and misuse considerations}
\label{sec:safety-misuse}

Systems optimized for persuasive argumentation can be dual-use; we address misuse risks, affected groups, and mitigations in the Adverse Impacts statement (Endmatter).

\subsection{Future work}
\label{sec:future-work}

First, human-subject evaluation of contestability as an interaction property: measuring time-to-challenge, challenge success rates, and perceived procedural fairness when participants attempt to locate evidence, challenge warrants, or compare perspective nodes.

Second, extending \foam{} with optimization methods while preserving contestability constraints, including training objectives that reward faithful warrant-evidence alignment rather than only persuasiveness.

% Section 8: Conclusion
% Last updated from: v2_2025-01-12.md

\section{Conclusion}
\label{sec:conclusion}

High-stakes deployments of LLM-based systems demand more than \emph{transparent-seeming} narratives; they require explanations that can be \emph{challenged, audited, and revised}. Recent evidence suggests that post-hoc ``reasoning traces'' are often not a reliable proxy for what drives model behavior: when a prompt-injected hint changes a model's answer, state-of-the-art reasoning models reveal that hint in their chain-of-thought only about \textbf{25--39\%} of the time, indicating substantial unfaithfulness of verbalized rationales to causal drivers of outputs~\cite{chen2025reasoning}. This paper contributes (1) \textbf{\foam{}}, a pluralistic deliberation architecture for explainability-and-contestability-by-design; (2) an \textbf{inspectable provenance mechanism} that makes sentence-level claims traceable to source spans and contestable at the level stakeholders actually dispute; and (3) an \textbf{audit-style empirical evaluation} in evidence-grounded policy debate generation. In a double-blind tournament of 66 cases, the \foam{}-based system achieves higher overall scores than expert-human and zero-shot baselines (Table~\ref{tab:main-results}) and dramatically higher perfect evidence validation rates (Table~\ref{tab:validation}), demonstrating that accountable generation can be simultaneously \emph{high-quality} and \emph{verifiable}.

For the FAccT community, the central implication is a practical shift from explanation-as-disclosure to \textbf{contestable explanations}: outputs whose \emph{claims, warrants, and evidence links} are explicit, inspectable, and designed to invite targeted challenge (\eg disputing a cited sentence, contesting a warrant, or requesting an alternative perspective node). This orientation is consistent with due-process motivations for a meaningful right to contest consequential automated decisions~\cite{kaminski2021right}. Where governance requires reason-giving that can withstand scrutiny, pluralistic deliberation plus verifiable provenance offers a concrete design pattern for building AI systems whose decisions can be examined, contested, and improved without relying on ``black-box'' rationalizations.


% === REQUIRED STATEMENTS FOR SUBMISSION ===
% Note: Author Contributions, Acknowledgements, Competing Interests,
% and Positionality Statement should be EXCLUDED for anonymous submission

\clearpage
\section*{Endmatter}

\begin{acks}
This work was supported by the National Science Foundation under SBIR Phase I Grant No. [GRANT\_NUMBER\_HERE].
\end{acks}

\subsection*{Generative AI Usage Statement}
This research investigates the use of large language models (LLMs) within a structured multi-agent deliberation framework. The \foam{} system uses LLMs as pipeline components: Claude 3 Haiku (Anthropic, March 2024) for primary generation, Claude 3.5 Sonnet (Anthropic, June 2024) for refinement phases, and Claude Opus 4 (Anthropic, January 2025) for tournament evaluation, as detailed in Section~\ref{sec:experimental-design}.

Regarding manuscript preparation: the paper text was drafted entirely by human authors. We used Claude 4.5 Opus and ChatGPT 4o solely for grammar checking, \LaTeX{} formatting assistance, and table alignment. No generative AI tools were used for idea generation, argument construction, literature review, experimental design, data analysis, or drafting of substantive prose. All intellectual contributions, claims, interpretations, and conclusions are the product of human judgment. The authors take full responsibility for the accuracy and integrity of this manuscript.

\subsection*{Ethical Considerations}
This work develops AI systems with persuasive capabilities, which raises dual-use concerns. We address these in Section~\ref{sec:limitations} and Section~\ref{sec:implications}, discussing safeguards including transparency requirements, evidence provenance constraints, and the deliberate choice to evaluate in a domain (competitive debate) with established norms for scrutinizing persuasive claims. The evaluation involved no human subjects; all baselines were drawn from publicly available debate materials or generated outputs.

\textbf{Evidence corpus licensing.} \foam{}'s evidence retrieval uses the OpenDebateEvidence corpus~\cite{roush2024opendebate}, released under CC-BY-4.0 for research purposes. Sentence-level provenance preserves attribution to original sources; we do not claim ownership of underlying evidence content.

\subsection*{Adverse Impacts Statement}
Systems that generate persuasive, evidence-grounded arguments could be misused for misinformation, manipulation, or to overwhelm human review capacity. Affected groups include decision-subjects in high-stakes domains and information consumers generally. We mitigate these risks through: (1) provenance requirements that make claims auditable; (2) evaluation in a domain with adversarial scrutiny norms; (3) architectural transparency (the deliberation trace is inspectable). Deployment in sensitive domains should include access controls, logging, human oversight, and institutional review processes.

% Bibliography
\bibliographystyle{ACM-Reference-Format}
\bibliography{references}

% Appendix
% Appendix: FOAM Implementation Details
% Technical specifications for the FOAM architecture

\appendix

\section{Core FOAM Components}
\label{sec:appendix-components}

\subsection{Perspective Nodes}
\label{sec:appendix-perspective}

A \textbf{Perspective Node} is a composite configuration that establishes the philosophical and methodological orientation for an agent throughout the deliberation process. Unlike a simple role assignment, perspective nodes encode multi-dimensional worldview parameters that constrain all downstream generation.

\textbf{Dimension Categories (32 total dimensions):}

\begin{itemize}
    \item \textbf{Debate Technique (11)}: resolution\_stance, argument\_architecture, negative\_strategy, organization\_structure, evidence\_integration, rhetorical\_framing, clash\_orientation, impact\_articulation, argument\_depth\_distribution, warrant\_density, theory\_deployment
    \item \textbf{Epistemological (2)}: epistemological\_stance (empirical positivism, constructivism, critical realism, standpoint theory, pragmatism), evidence\_hierarchy
    \item \textbf{Ethical/Impact (2)}: impact\_framework (utilitarian, deontological, virtue ethics, existential risk, structural violence), risk\_calculus
    \item \textbf{Strategic (1)}: strategic\_posture
    \item \textbf{Belief Paradigm (8)}: truth\_orientation, theism\_metaphysics, moral\_objectivity, human\_nature, source\_authority, free\_will\_stance, progress\_narrative, meaning\_of\_life
    \item \textbf{Policy Paradigm (8)}: fiscal\_orientation, market\_vs\_state, equity\_vs\_efficiency, social\_policy\_lens, global\_vs\_national, environmental\_stance, temporal\_horizon, governance\_style
\end{itemize}

\textbf{Coherence Scoring:} Perspective nodes include a coherence score (0.0--1.0) measuring internal consistency. The algorithm starts at 0.5, applies affinity bonuses ($+0.1 \times \text{strength}$) for compatible dimension pairs, applies incompatibility penalties ($-0.15 \times \text{severity}$) for conflicts, and clamps to [0.0, 1.0].

\subsection{Dialectical Refinement Protocol}
\label{sec:appendix-dialectical}

The dialectical refinement protocol implements iterative improvement through structured adversarial dialogue using a Proposer-Critic-Evaluator-Refiner loop.

\textbf{Configuration:}
\begin{itemize}
    \item \texttt{max\_iterations}: 5 (maximum refinement cycles)
    \item \texttt{convergence\_threshold}: Score variance threshold for early stopping
    \item \texttt{best\_of\_n}: 3 (candidates generated per role)
\end{itemize}

Convergence occurs when score variance falls below threshold, proposal achieves strong defense (score\_diff $> 5.0$), or maximum iterations reached.

\subsection{Flow Models (Deliberation Record)}
\label{sec:appendix-flow}

The deliberation record uses a hierarchical \textbf{Flow} model: Flow $\rightarrow$ FlowPage $\rightarrow$ FlowPageSpeech $\rightarrow$ Argument. Each Argument maintains explicit references to: the syllogism type structuring its logical form, the template node that allocated its word budget, evidence with sentence-level IDs, the guiding perspective, and any argument it rebuts.

\section{Pipeline Implementation}
\label{sec:appendix-pipeline}

\subsection{Five-Phase Generation Pipeline}

\textbf{Phase 1: Perspective Assignment.} Generate or select a PerspectiveNode, validate coherence, persist for downstream constraint enforcement.

\textbf{Phase 2: Plan Generation \& Refinement.} Generate 4 candidate policy positions, select most promising, apply dialectical refinement (minimum 3 iterations), conduct targeted web research.

\textbf{Phase 3: Template Tree Traversal.} Navigate hierarchical decision tree, allocate word budgets across syllogism types, generate TemplateTraversal objects for each leaf node.

\textbf{Phase 4: Research \& Evidence Gathering.} Query vector database (OpenDebateEvidence, $\sim$85k cards), conduct web research, apply sentence-level provenance, validate quotes against source fulltext.

\textbf{Phase 5: Compilation.} Assemble syllogisms in proper order, verify perspective consistency, validate evidence-claim alignment, output complete artifact.

\subsection{Typed Syllogisms}

FOAM enforces logical validity through \textbf{17 typed syllogisms}:

\begin{table}[h]
\centering
\small
\begin{tabular}{lll}
\toprule
\textbf{Type} & \textbf{Required Components} & \textbf{Context} \\
\midrule
advantage & uniqueness, link, internal\_link, impact & Affirmative benefits \\
inherency & barrier\_type, current\_status, barriers & Why status quo fails \\
solvency & mechanism, actor\_capability, effectiveness & How plan works \\
disadvantage & uniqueness, link, impact & Negative harms \\
counterplan & text, competition, net\_benefit & Alternative policy \\
topicality & interpretation, violation, standards, voter & Definitions \\
kritik & link, impact, alternative & Systemic critique \\
case\_turn & target, direction, impact & Flip aff argument \\
rebuttal & target, response\_type, warrant & Direct refutation \\
framework & interpretation, standards & Evaluative lens \\
\bottomrule
\end{tabular}
\caption{Selected typed syllogisms (10 of 17 shown)}
\label{tab:syllogisms}
\end{table}

\subsection{Template Tree Traversal}
\label{sec:appendix-template-tree}

The template tree is a hierarchical decision structure guiding argument generation and resource allocation. Each path from root to leaf represents a complete argument specification with word budget.

\textbf{Node Types:}
\begin{itemize}
    \item \texttt{root}: Entry point for debate format (e.g., ``Policy Debate'')
    \item \texttt{speech}: Speech type container (e.g., ``1AC'', ``1NC'')
    \item \texttt{branch}: Strategic decision point (e.g., ``Traditional'' vs ``Critical'')
    \item \texttt{leaf}: Terminal argument specification (e.g., ``Economic Impact'')
    \item \texttt{meta}: Cross-cutting template groups
\end{itemize}

\textbf{Example: Traditional 1AC Template Tree}

\begin{verbatim}
Policy Debate (root)
+-- 1AC (speech, 1300 words)
    +-- Plan Text (50 words)
    +-- Inherency (150 words, syllogism=inherency)
    |   +-- Structural Barrier (75 words)
    |   +-- Current Status (75 words)
    +-- Solvency (200 words, syllogism=solvency)
    |   +-- Mechanism (100 words)
    |   +-- Actor Capability (100 words)
    +-- Advantages (900 words)
        +-- Economic (450 words, syllogism=advantage)
        |   +-- Uniqueness (100 words)
        |   +-- Link (100 words)
        |   +-- Internal Link (100 words)
        |   +-- Impact (150 words)
        +-- Security (450 words, syllogism=advantage)
            +-- Uniqueness (100 words)
            +-- Link (100 words)
            +-- Internal Link (100 words)
            +-- Impact (150 words)
\end{verbatim}

\textbf{Traversal Process:} (1) Start at root and load debate format template; (2) Select speech type (1AC); (3) At each branch, LLM evaluates choice prompt based on perspective constraints, plan specifics, and strategic goals; (4) At leaves, generate TemplateTraversal objects recording the full path, word budget, research order, and syllogism type.

\textbf{Word Budget Validation:} Parent budget equals sum of children budgets. Minimum allocations enforced per syllogism component (e.g., impact $\geq$ 30\% of advantage). Overruns trigger automatic condensation.

\textbf{Dynamic Generation:} When existing templates lack an appropriate path, the system generates new TemplateNodes, mounts them to existing branches, propagates word budget, and continues traversal.

\subsection{Sentence-Level Provenance}
\label{sec:appendix-provenance}

The sentence-level provenance system prevents hallucination by constraining LLM outputs to reference existing text rather than reproduce it.

\textbf{Process:} (1) \textit{Indexing}: Each sentence receives a unique ID; (2) \textit{Selection}: LLM outputs sentence IDs rather than quoted text; (3) \textit{Assembly}: System retrieves actual sentences by ID; (4) \textit{Validation}: QuoteValidator confirms text exists in source with similarity threshold of 0.85.

\textbf{Match Classification:} \texttt{exact} (verbatim match, score 1.0), \texttt{partial} (substring match, 0.7--0.9), \texttt{paraphrase} (semantic match, 0.5--0.7), \texttt{not\_found} (no match, 0.0).

\section{Evaluation Methodology}
\label{sec:appendix-evaluation}

\subsection{Tournament Dataset}

The evaluation corpus consisted of 66 first affirmative constructive (1AC) cases:

\begin{table}[h]
\centering
\begin{tabular}{lcp{5cm}}
\toprule
\textbf{Source} & \textbf{N} & \textbf{Description} \\
\midrule
FOAM System & 22 & Generated with Claude Haiku 3 (primary) / Sonnet 3.5 (refinement) \\
Human Expert & 23 & Dartmouth, Georgetown, Michigan, Emory debate camps \\
Zero-Shot AI & 21 & Gemini, Claude, ChatGPT, Grok with deep research \\
\bottomrule
\end{tabular}
\caption{Tournament dataset composition}
\label{tab:dataset}
\end{table}

\textbf{Model Separation:} FOAM outputs were generated using Claude Haiku 3 (primary) and Sonnet 3.5 (refinement). Tournament judging was performed by Claude Opus 4, ensuring separation between generation and evaluation models to prevent self-enhancement bias.

\subsection{Tournament Protocol}

\textbf{Anonymization:} All cases assigned unique IDs (e.g., ``Case\_001''); metadata and formatting stripped; origin hidden from judges.

\textbf{Bracket Structure:} Modified Swiss-system with double elimination; initial grouping by strategic approach (Traditional, Kritik, Soft-Left); head-to-head evaluation in groups of 2--3; top 50\% advance per group; statistical ties (within 2.0 points) resolved by evidence validation.

\subsection{Judging Criteria}

Evaluation by Claude Opus 4 across five weighted dimensions:

\begin{table}[h]
\centering
\begin{tabular}{lcp{4.5cm}}
\toprule
\textbf{Dimension} & \textbf{Weight} & \textbf{Components} \\
\midrule
Argumentation Strength & 25\% & Logical consistency, warrant quality, impact development \\
Evidence Quality & 25\% & Source authenticity, validation scores \\
Strategic Coherence & 20\% & Internal consistency, preemptive handling \\
Innovation & 15\% & Novel arguments, differentiation \\
Competitive Viability & 15\% & Practical success potential \\
\bottomrule
\end{tabular}
\caption{Evaluation rubric dimensions}
\label{tab:rubric}
\end{table}

\subsection{Evidence Validation Results}

\textbf{Perfect Validation Rate} measures percentage of cases where ALL cited evidence achieves exact or partial match:

\begin{table}[h]
\centering
\begin{tabular}{lcc}
\toprule
\textbf{Source} & \textbf{Evidence Score} & \textbf{Perfect Validation} \\
\midrule
FOAM System & 86.7 & 76.2\% \\
Human Expert & 56.9 & 8.7\% \\
Zero-Shot AI & 27.1 & 0.0\% \\
\bottomrule
\end{tabular}
\caption{Evidence quality and validation rates}
\label{tab:validation-appendix}
\end{table}


\end{document}
