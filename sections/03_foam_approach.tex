% Section 3: FOAM Approach
% Last updated from: v1_2025-01-12_baseline.md

\section{FOAM approach: pluralistic architecture for explainability and contestability}
\label{sec:foam-approach}

\subsection{Design goals and accountability threat model}
\label{sec:design-goals}

Building on the accountability requirements in Section~\ref{sec:related-work}, we treat \emph{explainability} and \emph{contestability} as properties of an \textbf{epistemic process}, not a post-hoc narrative from a monolithic model. Concretely, we introduce \textbf{\foam{} (Framework for Openly Augmented Mediation)}: a pluralistic, multi-agent architecture that produces an answer \emph{plus} a structured record of how that answer was stress-tested, revised, and synthesized. \foam{} is organized around three primitives:
\begin{enumerate}
    \item \emph{Differentiated agents} parameterized by explicit and persistent stance data structures at test time,
    \item \emph{Deliberative protocols} that force critique and revision, and
    \item \emph{Sublation} operators that synthesize without erasing disagreement.
\end{enumerate}

Figure~\ref{fig:system-overview} provides a system overview: stance seeding $\rightarrow$ local drafting $\rightarrow$ cross-examination $\rightarrow$ evaluation $\rightarrow$ revision/sublation $\rightarrow$ accountability artifacts.

\begin{figure}[htbp]
\centering
\includegraphics[width=\columnwidth]{figures/FOAM_system_overview.png}
\caption{\foam{} system architecture. Differentiated agents with explicit perspective nodes engage in deliberative protocols (stance seeding, local drafting, cross-examination, evaluation) that produce accountability artifacts including a consensus core, conditional claims, and dissent memo. The mediation graph tracks claims, objections, and revisions throughout the process.}
\Description{Architecture diagram showing the FOAM system with six main processing stages (Differentiated Agents, Stance Seeding, Local Drafting, Cross-Examination, Evaluation, Sublation Operator) connected by arrows, with supporting components including Perspective Nodes feeding into agents, a Mediation Graph capturing deliberation, and three output types: Consensus Core, Conditional Claims, and Dissent Memo.}
\label{fig:system-overview}
\end{figure}

Our threat model is accountability-centric: we assume that base generative models can (a) produce fluent but false claims and fabricated or misattributed support (``hallucination''), (b) rationalize decisions after the fact, (c) collapse multiple stakeholder perspectives into a single dominant frame, and (d) bury value tradeoffs inside unstructured prose such that stakeholders cannot identify \emph{what}, precisely, to challenge. These failure modes do not require adversarial intent; they are well documented in contemporary NLP systems and can persist even under strong prompting~\cite{ji2023survey}. \foam{}'s core design choice is therefore to make \emph{points of potential failure} explicitly addressable: disagreements are surfaced rather than smoothed, objections are represented as first-class objects, and synthesis is constrained to preserve traceability from contested premises to final recommendations.

\subsection{Differentiated agents via explicit perspective and stance representation}
\label{sec:differentiated-agents}

\foam{} begins by instantiating a small set of agents ($n$ chosen by stakes and time budget), each assigned an explicit data structure represented as a \emph{Perspective Node} and stored in a vector database that encodes \emph{who the agent is meant to be epistemically}---its domain role, value priorities, and reasoning schema. This implements ``situated'' explanation in a directly auditable way: instead of implicitly claiming neutrality, the system discloses positions and thereby enables critique of the \emph{perspective selection} itself~\cite{haraway1988situated}. In \foam{}, perspective nodes are not just labels; they are operational constraints that shape what evidence is considered legitimate, which impacts are foregrounded, and which argument schemes are preferred.

Practically, we treat a perspective node as a structured record with three minimum components:
\begin{enumerate}
    \item \textbf{Role} (\eg regulator, clinician, affected community advocate),
    \item \textbf{Normative weighting} (\eg safety vs autonomy vs distributive equity), and
    \item \textbf{Epistemic policy} (\eg what counts as acceptable support; how uncertainty must be qualified).
\end{enumerate}

During deliberation, \foam{} enforces \emph{stance coherence}: if an agent's generated warrants or qualifiers contradict its declared stance, the system requests revision or flags the inconsistency for downstream inspection. This is the anti-``performative pluralism'' mechanism: pluralism is only accountability-relevant if the system can show (and users can contest) whether distinct perspectives were actually maintained rather than rhetorically simulated. For high-stakes governance applications, perspective dimensions should be restricted to institutionally defensible attributes---stakeholder role, domain expertise, value priorities, evidentiary standards---rather than the full persona parameterization used in our debate instantiation. Who selects perspectives, and by what legitimation process, is itself a governance question that the technical architecture surfaces but does not resolve.

Perspective nodes also make \textbf{second-order contestation} practical: stakeholders can dispute not only the system's conclusion, but the \emph{legitimacy of the value and perspective configuration} that produced it (\eg ``Why is utilitarian cost-effectiveness even in scope here?''). This matters because pluralistic systems can otherwise ``value-wash'' by claiming inclusivity while quietly privileging one evaluative frame. \foam{} makes the stance set an explicit input and therefore a target for governance and oversight; this aligns with work arguing that legitimacy depends on making value choices and their selection procedures contestable~\cite{kasirzadeh2024plurality}. In deployment terms, this means \foam{} can be rerun with (i) added perspectives, (ii) reweighted value priorities, or (iii) altered evidentiary rules, producing \emph{comparative, contestable} outcomes rather than a single authoritative verdict.

\subsection{Deliberative protocol: dialectical refinement and mediation trace}
\label{sec:deliberative-protocol}

\foam{}'s deliberation is implemented as a \textbf{mediation loop}:
\begin{enumerate}
    \item \emph{Seeding} (instantiate agents + perspective nodes),
    \item \emph{Local drafting} (agents generate independent proposals),
    \item \emph{Cross-examination} (agents issue structured objections and targeted questions),
    \item \emph{Evaluation} (a judge/jury component scores draft--objection pairs against criteria), and
    \item \emph{Revision + synthesis} (agents revise and a sublation operator composes the provisional output).
\end{enumerate}

The accountability point is not that deliberation guarantees truth; it is that \textbf{deliberation guarantees structured opportunities to find and localize error}, and then to record what happened when error was raised.

Cross-examination produces a \textbf{mediation graph}: a structured trace that links \emph{which agent} made \emph{which claim}, what objections were raised (\eg missing evidence, value conflict, logical gap), how the claim was revised, and which surviving claims contributed to the synthesis. This is the audit primitive: contestation requires that stakeholders can point to \emph{the specific node} where they disagree and see what depended on it. As an interoperability target, the mediation trace can be expressed using standard provenance representations (\eg PROV-O) so that downstream tools can query ``what influenced what'' across a run~\cite{lebo2013prov}.

\subsection{Sublation: synthesis without erasure}
\label{sec:sublation}

After critique and revision, \foam{} applies a \textbf{sublation operator}: a synthesis step intended to preserve what is valuable in competing positions while explicitly retaining unresolved tensions. In \foam{}, sublation is not a rhetorical flourish; it is a concrete rule: synthesis is disallowed from silently discarding objections that were scored as material or from collapsing incompatible value frames into an unmarked compromise.

Sublation emits three artifacts: \textbf{a consensus core} (claims surviving cross-stance critique), \textbf{conditional claims} (branching on unresolved value priorities), and a \textbf{dissent memo} (recording conflicts and contested premises).

\subsection{Inspectable argument structure: Toulmin decomposition and typed syllogisms}
\label{sec:argument-structure}

To make contestation actionable, \foam{} constrains agent outputs into an \textbf{inspectable argument structure} rather than free-form prose. We adopt Toulmin-style decomposition---claim, grounds, warrant, backing, qualifier, rebuttal---because it maps naturally to ``what can be challenged'': stakeholders can contest evidence (grounds), the inferential link (warrant), scope conditions (qualifier), or missing counterevidence (rebuttal)~\cite{toulmin1958uses}. This structure also aligns with prior work connecting computational argumentation to explainable AI, where explanations are made more useful by exposing structured reasons and counterreasons rather than only surface-level narratives~\cite{vassiliades2021argumentation}.

\foam{} additionally employs \textbf{typed syllogisms}---domain-relevant argument templates that enforce completeness (\eg in policy debate: Advantage = Uniqueness + Link + Impact; Disadvantage = Uniqueness + Link + Impact; Kritik = Link + Impact + Alternative). These structures are standard in competitive policy debate pedagogy and make dependencies explicit for non-expert audiences~\cite{snider2008code}. In \foam{}, typed syllogisms function as contestability scaffolds: if a stakeholder disputes the conclusion, the system can point to the \emph{specific missing or weak component} (\eg ``impact evidence absent'' or ``link warrant unsupported''), and the mediation graph can show whether that component was ever raised in critique and why it survived. The result is a system where ``challenge'' is not a vague request to ``explain more,'' but a targeted operation on a specific argumentative component with traceable upstream dependencies.

Template tree traversal operationalizes structural contestability. At each branch point, the system records which template was selected (\eg ``traditional 1AC with 3 advantages'' vs.\ ``kritik with alternative''), what resource allocation was applied, and whether any novel templates were generated. This trace enables a distinct class of challenges: stakeholders can dispute not only \emph{what} claims were made, \emph{but why the argumentative structure took this form rather than another.} For instance, a reviewer might contest that a utilitarian impact calculus was chosen when the underlying values favor a rights-based framing---and the template selection trace makes this challenge actionable. Furthermore, unlike traditional chain-of-thought reasoning where reasoning and response are interwoven and in some cases reasoning is not always a reliable indicator for why outputs occurred, the template tree traversal process is a discrete step occurring prior and serving as a foundational infrastructure to drafting.
